
\documentclass{beamer}
\usepackage[polish]{babel}
\usepackage[utf8]{inputenc}
\usepackage[T1]{fontenc}

\usetheme{Montpellier}
\beamersetuncovermixins{\opaqueness<1>{25}}{\opaqueness<2->{15}}

\usecolortheme{dove}
\begin{document}
\title{Zastosowania i perspektywy interfejsu mózg-komputer}
\author{Filip Chudy}
\date{\today} 


\begin{frame}
\titlepage
\end{frame}

% potrzebne grafiki: EEG, dane EEG, wykres różnych częstotliwości fal
% potrzebne grafiki: regresja
% do zwymiarowania jeszcze

% tu bedzie wahadlo magnetyczne
\section{Sterowanie myślami}

\subsection{Inspiracje}
\begin{frame} \frametitle{Gwiezdne Wojny}
\includegraphics[width=\textwidth,height=0.8\textheight,keepaspectratio]{sw.jpg}
\end{frame}

\begin{frame} \frametitle{Transcendencja}
\includegraphics[width=\textwidth,height=0.8\textheight,keepaspectratio]{trans.jpg}
\end{frame}

\subsection{Realne możliwości}
\begin{frame} \frametitle{Elektroencefalografia (EEG)}
\begin{columns}
\begin{column}{5cm}
\includegraphics[width=\textwidth,height=0.8\textheight,keepaspectratio]{eeg.jpg}
\end{column}
\begin{column}{5cm}
Naturalnym pomysłem na nieinwazyjny odczyt sygnałów mózgowych jest EEG.
\end{column}
\end{columns}
\end{frame}

\begin{frame} \frametitle{Elektroencefalografia (EEG)}
\begin{columns}
\begin{column}{5cm}
\includegraphics[width=\textwidth,height=0.8\textheight,keepaspectratio]{eeg_wyniki.png}
\end{column}
\begin{column}{5cm}
Dane są sygnałem zbieranym przez elektrody rozmieszczone na głowie.
\end{column}
\end{columns}
\end{frame}

\begin{frame} \frametitle{Emotiv}
\begin{columns}
\begin{column}{5cm}
\includegraphics[width=\textwidth,height=0.8\textheight,keepaspectratio]{emotiv.jpg}
\end{column}
\begin{column}{5cm}	
Wykorzystane zostało (zhakowane) urządzenie Emotiv.
\end{column}
\end{columns}
\end{frame}

\section{Co zrobić z sygnałami?}
\subsection{Osiągnięcia medycyny}
\begin{frame} \frametitle{Analiza Hjortha}
\begin{columns}
\begin{column}{5cm}
\includegraphics[width=\textwidth,height=0.8\textheight,keepaspectratio]{hjorth.jpg}
\end{column}
\begin{column}{5cm}
Wyliczanie z sygnału trzech wskaźników: Activity, Mobility, Complexity.
Skutecznie wykrywa anomalie, ale słabo się spisuje w klasyfikacji odczytów z osoby zdrowej.
\end{column}
\end{columns} 
\end{frame}

\begin{frame} \frametitle{Podział fal mózgowych}
\begin{columns}
\begin{column}{5cm}
\includegraphics[width=\textwidth,height=0.8\textheight,keepaspectratio]{bw.jpg}
\end{column}
\begin{column}{5cm}
\includegraphics[width=\textwidth,height=0.8\textheight,keepaspectratio]{bw_freqs.png}
\end{column}
\end{columns} 
\end{frame}

\subsection{Przetwarzanie sygnałów}
\begin{frame} \frametitle{Filtry}
\includegraphics[width=\textwidth,height=0.8\textheight,keepaspectratio]{filter3.png}
\end{frame}

\begin{frame} \frametitle{FFT}
\includegraphics[width=\textwidth,height=0.8\textheight,keepaspectratio]{fourier.jpg}
\end{frame}

\section{Jak klasyfikować dane}
\subsection{Klasyczne metody}
\begin{frame} \frametitle{Regresja}
Prosty model, ale wrażliwość na pojedyncze dane z dużym odchyleniem utrudnia jego użycie.
\end{frame}

\begin{frame} \frametitle{Sieci neuronowe}
\begin{columns}
\begin{column}{5cm}
 \includegraphics[width=\textwidth,height=0.8\textheight,keepaspectratio]{Image676.png}
\end{column}

 \begin{column}{5cm}
\begin{itemize}
 \item model nieliniowy, więc silniejszy od regresji
 \pause \item trzeba ustalić strukturę sieci
 \pause \item duży wektor parametrów -- trudniejsza optymalizacja 
\end{itemize}  
 \end{column}

\end{columns}


\end{frame}

\begin{frame} \frametitle{Sieci neuronowe}
\begin{columns}
\begin{column}{5cm}
\includegraphics[width=\textwidth,height=0.8\textheight,keepaspectratio]{gorki.png}
\end{column}
\begin{column}{5cm}
Nie mamy gwarancji, że parametry zostaną zoptymalizowane do ekstremum globalnego.
\end{column}
\end{columns}
\end{frame}

\begin{frame} \frametitle{Support Vector Machines}
\begin{columns}
\begin{column}{5cm}
\includegraphics[width=\textwidth,height=0.8\textheight,keepaspectratio]{image_3.jpg}
\end{column}
\begin{column}{5cm}
Programowanie kwadratowe pozwala osiągnąć minimum globalne.
Zarówno w wariancie bezbłędnym...
\end{column} 
\end{columns}
\end{frame}

\begin{frame} \frametitle{Support Vector Machines}
\begin{columns}
\begin{column}{5cm}
\includegraphics[width=\textwidth,height=0.8\textheight,keepaspectratio]{vgv4.png}
\end{column}
\begin{column}{5cm}
Programowanie kwadratowe pozwala osiągnąć minimum globalne.
...jak i pozwalającym na błędy.
\end{column} 
\end{columns}
\end{frame}

\subsection{Wybrana metoda}

\begin{frame} \frametitle{Support Vector Machines}
Bardziej elastyczne od regresji, wygodniejsze od sieci neuronowych.
\end{frame}

\begin{frame} \frametitle{Support Vector Machines}
Parametr $C$ został dobrany przy pomocy algorytmu ewolucyjnego...

\pause ... czyli kilkunastu godzin obliczeń na 16 rdzeniach.
\end{frame}

\begin{frame} \frametitle{Jakość działania}
$3$ grupy

$8$ sekund danych treningowych dla każdej z grupy

Skuteczność: ok. $90\%$
\end{frame}


\begin{frame} \frametitle{Jakość działania}
Pomyłki między L a R występują bardzo rzadko.

System jest sceptyczny -- nie produkuje zbyt wielu fałszywych alarmów.
\end{frame}

\section{Perspektywy i ograniczenia}
\begin{frame} \frametitle{Perspektywy}
System sprawuje się dobrze przy zastosowaniach nie wymagających szybkiej reakcji i klasyfikacji.
\end{frame}

\begin{frame} \frametitle{Ograniczenia -- urządzenie}
Elektrody muszą być regularnie nawilżane.

Fryzura.

Po $45$ minutach boli głowa.
\end{frame}

\begin{frame} \frametitle{Ograniczenia -- człowiek}
\includegraphics[width=\textwidth,height=0.8\textheight,keepaspectratio]{distraction.jpg}
\end{frame}

\end{document}